\documentclass[]{article}
\usepackage{lmodern}
\usepackage{amssymb,amsmath}
\usepackage{ifxetex,ifluatex}
\usepackage{fixltx2e} % provides \textsubscript
\ifnum 0\ifxetex 1\fi\ifluatex 1\fi=0 % if pdftex
  \usepackage[T1]{fontenc}
  \usepackage[utf8]{inputenc}
\else % if luatex or xelatex
  \ifxetex
    \usepackage{mathspec}
  \else
    \usepackage{fontspec}
  \fi
  \defaultfontfeatures{Ligatures=TeX,Scale=MatchLowercase}
\fi
% use upquote if available, for straight quotes in verbatim environments
\IfFileExists{upquote.sty}{\usepackage{upquote}}{}
% use microtype if available
\IfFileExists{microtype.sty}{%
\usepackage[]{microtype}
\UseMicrotypeSet[protrusion]{basicmath} % disable protrusion for tt fonts
}{}
\PassOptionsToPackage{hyphens}{url} % url is loaded by hyperref
\usepackage[unicode=true]{hyperref}
\hypersetup{
            pdftitle={MXB262 Workshop week 3: Effective Visual Communication},
            pdfauthor={Dr Kate Helmstedt},
            pdfborder={0 0 0},
            breaklinks=true}
\urlstyle{same}  % don't use monospace font for urls
\usepackage[margin=1in]{geometry}
\usepackage{graphicx,grffile}
\makeatletter
\def\maxwidth{\ifdim\Gin@nat@width>\linewidth\linewidth\else\Gin@nat@width\fi}
\def\maxheight{\ifdim\Gin@nat@height>\textheight\textheight\else\Gin@nat@height\fi}
\makeatother
% Scale images if necessary, so that they will not overflow the page
% margins by default, and it is still possible to overwrite the defaults
% using explicit options in \includegraphics[width, height, ...]{}
\setkeys{Gin}{width=\maxwidth,height=\maxheight,keepaspectratio}
\IfFileExists{parskip.sty}{%
\usepackage{parskip}
}{% else
\setlength{\parindent}{0pt}
\setlength{\parskip}{6pt plus 2pt minus 1pt}
}
\setlength{\emergencystretch}{3em}  % prevent overfull lines
\providecommand{\tightlist}{%
  \setlength{\itemsep}{0pt}\setlength{\parskip}{0pt}}
\setcounter{secnumdepth}{0}
% Redefines (sub)paragraphs to behave more like sections
\ifx\paragraph\undefined\else
\let\oldparagraph\paragraph
\renewcommand{\paragraph}[1]{\oldparagraph{#1}\mbox{}}
\fi
\ifx\subparagraph\undefined\else
\let\oldsubparagraph\subparagraph
\renewcommand{\subparagraph}[1]{\oldsubparagraph{#1}\mbox{}}
\fi

% set default figure placement to htbp
\makeatletter
\def\fps@figure{htbp}
\makeatother


\title{MXB262 Workshop week 3: Effective Visual Communication}
\author{Dr Kate Helmstedt}
\date{Semester 1, 2021}

\begin{document}
\maketitle

\subsection{This week}\label{this-week}

This week you will learn how to:

\begin{enumerate}
\def\labelenumi{\arabic{enumi}.}
\tightlist
\item
  Embed an image in a markdown file\\
\item
  Create a story board using the narrative approach to data
  visualisation\\
\item
  Submit an R markdown file for the Problem Solving Task assessment
  (\textbf{make sure you have a version that knits well before the due
  date even if it is relatively empty, so that you can post issues to
  Slack or come to the workshop if you need a hand knitting it for
  submission})
\item
  Start some initial data exploration in R -- we won't be visualising
  this today, but lots of the techniques you learn will be invaluable
  when we are exploring data using visualisation next week, and in your
  projects
\end{enumerate}

\subsection{1. Adding Images}\label{adding-images}

Since this is a visualisation unit, we will be using and examining many
different images -- and you will need to submit some in your Problem
Solving Tasks and Projects. There are two types of images we will be
using, and each is added to an R markdown document differently. - Most
frequently, we will be making new visualisations using R commands and
raw datasets. To make these images, we will add code chunks to the R
markdown documents that generate the figures. That way, the
visualisation is created when we knit the R markdown document. -
Sometimes, including this week, we will need to refer to and examine
existing images from other sources. These need to be added to the R
markdown documents. Use the line of code
\texttt{!{[}Optional\ Caption{]}(insert\ URL\ or\ file\ path)\{width=some\ percentage\ of\ the\ text\ width\ from\ 0\ to\ 100\}}
to embed images within the R markdown document. This does not need to be
added within a code chunk.

For example, the command
\texttt{!{[}What\textquotesingle{}s\ the\ deal\ with\ Leonardo\ DiCaprio{]}(https://i.redd.it/erdv3yn78d941.png)\{width=75\%\}}
prints one of my favourite visualisations a student found in a previous
year in this Problem Solving Task, which is being pulled from the reddit
website
\includegraphics[width=0.75000\textwidth]{https://i.redd.it/erdv3yn78d941.png}

To find the URL of an image online, right click on the image and select
``Coppy Image Address''.

To add an image from your computer instead, replace the URL with a file
path. If you choose to do this, please make sure you're submitting the
image file along with your code and your knitted version.

\subsection{2. Narrative visualisations and story
telling}\label{narrative-visualisations-and-story-telling}

In this week's lecture you learnt about:

\begin{enumerate}
\def\labelenumi{\alph{enumi}.}
\tightlist
\item
  \emph{Context elements of a story - Who/What/When/How}
\item
  \emph{Genres of Narrative Visualisation}
\end{enumerate}

\begin{figure}
\centering
\includegraphics{https://www.mdpi.com/information/information-09-00065/article_deploy/html/images/information-09-00065-g021-550.jpg}
\caption{Genres of Narrative Visualisation (Segel, 2010). Source:
\url{https://www.mdpi.com/2078-2489/9/3/65/htm}}
\end{figure}

\begin{enumerate}
\def\labelenumi{\alph{enumi}.}
\setcounter{enumi}{2}
\tightlist
\item
  \emph{Author vs Audience Driven Visualisations} (See lecture notes).
\end{enumerate}

\subsection{Story board Example 1}\label{story-board-example-1}

Your Problem Solving task this week will be to create a few storyboards
for different visualisations (including one you need to find yourself!).

We're going to run you through a couple of examples first, which use the
information you learned in the lecture this week.

\textbf{Title \& Link:}
\href{https://www.nytimes.com/interactive/2019/02/13/climate/cut-us-emissions-with-policies-from-other-countries.html}{How
fast emissions would reduce if other plans were adopted} -- take a look
at the link, it is a graphic that builds on itself to give context.

\textbf{Image:} This is just a static screenshot of an animated
visualisation, which you will need to include if you use any animated
visualisations. Please click through to see the whole animation before
exploring the storyboard below.

\begin{figure}
\centering
\includegraphics[width=0.75000\textwidth]{https://i1.wp.com/flowingdata.com/wp-content/uploads/2019/02/Cut-U.S.-emissions.png?w=2048\&ssl=1}
\caption{How fast emissions would reduce if other plans were adopted.
Sourced from:
\url{https://i1.wp.com/flowingdata.com/wp-content/uploads/2019/02/Cut-U.S.-emissions.png?w=2048\&ssl=1}}
\end{figure}

\paragraph{A. Describing the context}\label{a.-describing-the-context}

\textbf{Who is the audience or audiences?: }

Educated readers from the general public (i.e.~not necessarily
policy-makers or scientists) who are informed about the climate crisis,
and agree with the goal to reduce emissions.

\textbf{What is the action the visualisation is aiming for? Consider
each audience here if you have multiple:}

\begin{itemize}
\tightlist
\item
  Agree/learn/realise that even with strong action, emissions goals will
  be missed.
\item
  Act to reduce carbon emissions.
\item
  Realise that switching all electricity sources to zero-carbon energy
  sources will not be enough to reach the national goal.\\
\item
  Realise and agree that more drastic and urgent action is needed now.\\
\item
  Act to introduce policy that does a better job at reducing emissions
  to national goal.
\end{itemize}

\textbf{When can the communication happen, and what tools have been used
to suggest an order:}

Animated image allows multple shots at communication to build the
graphic. In each static graphic, multiple factors have been used to
suggest order: colour, line angle, colour intensity, and enclosure.

\textbf{How has the data been used to convey the action?:}\\
Line plot of current emissions, compared with line plot of projected
emissions if zero-carbon energy sources were used, compared with the
national goal in red. Text labelling has been used.

\paragraph{B. Genre}\label{b.-genre}

\textbf{Which of the seven genres listed above best describes the data
visualisation?}

Slide show.

\paragraph{C. Author-driven vs
Reader-driven}\label{c.-author-driven-vs-reader-driven}

\textbf{Where on the spectrum from author- to reader-driven is this
visualisation?}

Author-driven, since new information is revealed in the order the author
dictates.

\subsection{Story boardExample 2}\label{story-boardexample-2}

We will work progressively through this one together. This is not a
marked exercise, but I suggest you think about your answers to practice
before revealing some suggested answers by hovering your mouse over the
text. Thinking back to the Leonardo DiCaprio visualisation,
\includegraphics[width=0.75000\textwidth]{https://i.redd.it/erdv3yn78d941.png}

\paragraph{A. Describing the context}\label{a.-describing-the-context-1}

\textbf{Who is the audience or audiences?: }

 ``Some options include: Leonardo DiCaprio fans, People interested in
celebrity news''

**``What is the action the visualisation is aiming for? Consider each
audience here if you have multiple:**

 Some options include: Remember that Leonardo DiCaprio only dates
younger women, Agree that it is unusual to only date women below 25.

 Answers that are not correct include: Learn the names of Leonardo
DiCaprio's ex-girlfriends, Remember Leonardo DiCaprio's current age.
This is because although the visualisation does include this
information, it is not a key feature or takeaway of the image.

\textbf{When can the communication happen, and what tools have been used
to suggest an order:}

 All information is given at once in a static image.

 Colour has been used to differentiate between Leonardo DiCaprio's age
and the ages of the women he dates, suggesting that the ages of the
women should be read at the same time as one group, and then compared to
Leonardo. Order down the page suggests that the plot should be read
first, and then the years that correspond to each ex-partner should be
read later. The size of a small quote at the bottom (compared to the
text of the rest of the plot) suggest it should be read last, as an
additional piece of background information not critical to the story.

\textbf{How has the data been used to convey the action?:}

 A bar plot indicating women's ages, and a time series line plot
indicating Leonardo DiCaprio's age. These are aligned on the same axis,
suggesting they are compared at each point in time. Text labelling and
images have been used to annotated the timeline.

\paragraph{B. Genre}\label{b.-genre-1}

\textbf{Which of the seven genres listed above best describes the data
visualisation?:}

 Annotated chart.

\paragraph{C. Author-driven vs
Reader-driven}\label{c.-author-driven-vs-reader-driven-1}

\textbf{Where on the spectrum from author- to reader-driven is this
visualisation?:}

 Reader-driven as the information can be read in any order.

\subsection{TASK: Create a storyboard}\label{task-create-a-storyboard}

\begin{quote}
Task 1: Use these principles of narration and stort-telling to
back-engineer ONE story boards with all of the elements above from the
list of visualisations below. This will contribute to your Problem
Solving Task for submission.
\end{quote}

\begin{enumerate}
\def\labelenumi{\arabic{enumi}.}
\tightlist
\item
  \href{https://www.abc.net.au/news/2018-11-22/counting-the-cost-of-the-education-revolution/10495756}{Counting
  the cost of the education revolution}
\item
  \href{https://atlas.cancer.org.au/}{Australian Cancer Atlas} - launch
  the atlas to see the visualisation.\\
\item
  \href{https://earth.nullschool.net/\#current/wind/isobaric/1000hPa/orthographic=25.77,0.41,440}{Global
  wind map}\\
\item
  \href{https://www.behance.net/gallery/70033395/The-Most-Violent-Cities/}{The
  most violent cities in the world}\\
\item
  \href{https://www.axios.com/yes-there-really-is-a-lot-of-space-junk-930b166d-68c4-4803-9bd6-9e23514eb942.html}{Yes,
  there is a lot of space junk}\\
\item
  \href{https://showyourstripes.info/}{Climate stripes}
\end{enumerate}

\textbf{Visualisation Title \& Link:}

\textbf{Image:}

If you're grabbing this from the internet, make sure you right click on
the image and select (`Copy image address'). If it's an animation,
you'll need to take a screenshot and embed that filepath here. Please
include all images in your submission.

\subparagraph{A. Describing the
context}\label{a.-describing-the-context-2}

\textbf{Who is the audience or audiences?: }

\textbf{What is the action the visualisation is aiming for? Consider
each audience here:}

\textbf{When can the communication happen, and what tools have been used
to suggest an order:}

\textbf{How has the data been used to convey the action?:}

\subparagraph{B. Genre}\label{b.-genre-2}

\textbf{Which of the seven genres listed above best dsecribes the data
visualisation?}

\subparagraph{C. Author-driven vs
Reader-driven}\label{c.-author-driven-vs-reader-driven-2}

\textbf{Where on the spectrum from author- to reader-driven is this
visualisation?}

\subsection{TASK: R for Data Science}\label{task-r-for-data-science}

\begin{quote}
\begin{enumerate}
\def\labelenumi{\arabic{enumi}.}
\setcounter{enumi}{1}
\tightlist
\item
  Work through
  \href{https://r4ds.had.co.nz/data-visualisation.html}{Chapters 4
  (short) and 5 of R for Data Science}. Read and follow all the examples
  along in your own R console for:
\end{enumerate}
\end{quote}

\begin{itemize}
\tightlist
\item
  \href{https://r4ds.had.co.nz/workflow-basics.html}{Chapter 4}\\
\item
  \href{https://r4ds.had.co.nz/transform.html\#introduction-2}{Sections
  5.1}
\item
  \href{https://r4ds.had.co.nz/transform.html\#filter-rows-with-filter}{5.2
  (exercises 1-3)}
\item
  \href{https://r4ds.had.co.nz/transform.html\#arrange-rows-with-arrange}{5.3
  (and all exercises)}
\item
  \href{https://r4ds.had.co.nz/transform.html\#select}{5.4 (skip these
  exercises)}
\item
  \href{https://r4ds.had.co.nz/transform.html\#add-new-variables-with-mutate}{5.5
  (exercises 1-5)}
\item
  \href{https://r4ds.had.co.nz/transform.html\#grouped-summaries-with-summarise}{5.6.1
  and 5.6.2.}
\end{itemize}

You can go back to the exercises you skipped outside of class time,
although they aren't critical at this stage of the unit.

\emph{Reminder:} You may want to just play around in R and get things to
work before copying them into R markdown files for your PST. To do this,
or to follow along these examples in the textbook, you can type straight
into the Console. That is quick and easy, but doesn't let you save your
code. You'll need to load the libraries we need into the Console in that
case.

If you want to save R code but don't necessarily want it in the .Rmd
file for your assessment, click File \textgreater{} New file
\textgreater{} R script, and you can type away in there -- and save it
as a separate file (similar to an m file in matlab). To run code from an
R file, highlight the section you want to run and click `Run' above, or
just copy paste it into the console. That way you can edit multiple
lines at once, and save the whole script. This is a really useful option
for keeping useful commends from these workshops all in one place. At
the beginning of each R script, it is good practice to add the
\texttt{library()} lines that load the packages you need -- you will
thank yourself later!

Remember that if you get an error when you try to load a package, you
might not have installed it on that computer yet.
\texttt{install.packages("package\_name\_that\_you\_want\_to\_use\_goes\_here\_dont\_forget\_the\_quotation\_marks\_every\_beginner\_does")}
is your trusty command to install the packages you need.

\end{document}
