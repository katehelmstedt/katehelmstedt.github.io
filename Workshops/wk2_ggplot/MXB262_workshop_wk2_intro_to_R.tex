\documentclass[]{article}
\usepackage{lmodern}
\usepackage{amssymb,amsmath}
\usepackage{ifxetex,ifluatex}
\usepackage{fixltx2e} % provides \textsubscript
\ifnum 0\ifxetex 1\fi\ifluatex 1\fi=0 % if pdftex
  \usepackage[T1]{fontenc}
  \usepackage[utf8]{inputenc}
\else % if luatex or xelatex
  \ifxetex
    \usepackage{mathspec}
  \else
    \usepackage{fontspec}
  \fi
  \defaultfontfeatures{Ligatures=TeX,Scale=MatchLowercase}
\fi
% use upquote if available, for straight quotes in verbatim environments
\IfFileExists{upquote.sty}{\usepackage{upquote}}{}
% use microtype if available
\IfFileExists{microtype.sty}{%
\usepackage{microtype}
\UseMicrotypeSet[protrusion]{basicmath} % disable protrusion for tt fonts
}{}
\usepackage[margin=1in]{geometry}
\usepackage{hyperref}
\hypersetup{unicode=true,
            pdftitle={MXB262 Workshop week 2: Introduction to R, ggplot2, and Rmarkdown},
            pdfauthor={Dr Kate Helmstedt},
            pdfborder={0 0 0},
            breaklinks=true}
\urlstyle{same}  % don't use monospace font for urls
\usepackage{color}
\usepackage{fancyvrb}
\newcommand{\VerbBar}{|}
\newcommand{\VERB}{\Verb[commandchars=\\\{\}]}
\DefineVerbatimEnvironment{Highlighting}{Verbatim}{commandchars=\\\{\}}
% Add ',fontsize=\small' for more characters per line
\usepackage{framed}
\definecolor{shadecolor}{RGB}{248,248,248}
\newenvironment{Shaded}{\begin{snugshade}}{\end{snugshade}}
\newcommand{\AlertTok}[1]{\textcolor[rgb]{0.94,0.16,0.16}{#1}}
\newcommand{\AnnotationTok}[1]{\textcolor[rgb]{0.56,0.35,0.01}{\textbf{\textit{#1}}}}
\newcommand{\AttributeTok}[1]{\textcolor[rgb]{0.77,0.63,0.00}{#1}}
\newcommand{\BaseNTok}[1]{\textcolor[rgb]{0.00,0.00,0.81}{#1}}
\newcommand{\BuiltInTok}[1]{#1}
\newcommand{\CharTok}[1]{\textcolor[rgb]{0.31,0.60,0.02}{#1}}
\newcommand{\CommentTok}[1]{\textcolor[rgb]{0.56,0.35,0.01}{\textit{#1}}}
\newcommand{\CommentVarTok}[1]{\textcolor[rgb]{0.56,0.35,0.01}{\textbf{\textit{#1}}}}
\newcommand{\ConstantTok}[1]{\textcolor[rgb]{0.00,0.00,0.00}{#1}}
\newcommand{\ControlFlowTok}[1]{\textcolor[rgb]{0.13,0.29,0.53}{\textbf{#1}}}
\newcommand{\DataTypeTok}[1]{\textcolor[rgb]{0.13,0.29,0.53}{#1}}
\newcommand{\DecValTok}[1]{\textcolor[rgb]{0.00,0.00,0.81}{#1}}
\newcommand{\DocumentationTok}[1]{\textcolor[rgb]{0.56,0.35,0.01}{\textbf{\textit{#1}}}}
\newcommand{\ErrorTok}[1]{\textcolor[rgb]{0.64,0.00,0.00}{\textbf{#1}}}
\newcommand{\ExtensionTok}[1]{#1}
\newcommand{\FloatTok}[1]{\textcolor[rgb]{0.00,0.00,0.81}{#1}}
\newcommand{\FunctionTok}[1]{\textcolor[rgb]{0.00,0.00,0.00}{#1}}
\newcommand{\ImportTok}[1]{#1}
\newcommand{\InformationTok}[1]{\textcolor[rgb]{0.56,0.35,0.01}{\textbf{\textit{#1}}}}
\newcommand{\KeywordTok}[1]{\textcolor[rgb]{0.13,0.29,0.53}{\textbf{#1}}}
\newcommand{\NormalTok}[1]{#1}
\newcommand{\OperatorTok}[1]{\textcolor[rgb]{0.81,0.36,0.00}{\textbf{#1}}}
\newcommand{\OtherTok}[1]{\textcolor[rgb]{0.56,0.35,0.01}{#1}}
\newcommand{\PreprocessorTok}[1]{\textcolor[rgb]{0.56,0.35,0.01}{\textit{#1}}}
\newcommand{\RegionMarkerTok}[1]{#1}
\newcommand{\SpecialCharTok}[1]{\textcolor[rgb]{0.00,0.00,0.00}{#1}}
\newcommand{\SpecialStringTok}[1]{\textcolor[rgb]{0.31,0.60,0.02}{#1}}
\newcommand{\StringTok}[1]{\textcolor[rgb]{0.31,0.60,0.02}{#1}}
\newcommand{\VariableTok}[1]{\textcolor[rgb]{0.00,0.00,0.00}{#1}}
\newcommand{\VerbatimStringTok}[1]{\textcolor[rgb]{0.31,0.60,0.02}{#1}}
\newcommand{\WarningTok}[1]{\textcolor[rgb]{0.56,0.35,0.01}{\textbf{\textit{#1}}}}
\usepackage{longtable,booktabs}
\usepackage{graphicx,grffile}
\makeatletter
\def\maxwidth{\ifdim\Gin@nat@width>\linewidth\linewidth\else\Gin@nat@width\fi}
\def\maxheight{\ifdim\Gin@nat@height>\textheight\textheight\else\Gin@nat@height\fi}
\makeatother
% Scale images if necessary, so that they will not overflow the page
% margins by default, and it is still possible to overwrite the defaults
% using explicit options in \includegraphics[width, height, ...]{}
\setkeys{Gin}{width=\maxwidth,height=\maxheight,keepaspectratio}
\IfFileExists{parskip.sty}{%
\usepackage{parskip}
}{% else
\setlength{\parindent}{0pt}
\setlength{\parskip}{6pt plus 2pt minus 1pt}
}
\setlength{\emergencystretch}{3em}  % prevent overfull lines
\providecommand{\tightlist}{%
  \setlength{\itemsep}{0pt}\setlength{\parskip}{0pt}}
\setcounter{secnumdepth}{0}
% Redefines (sub)paragraphs to behave more like sections
\ifx\paragraph\undefined\else
\let\oldparagraph\paragraph
\renewcommand{\paragraph}[1]{\oldparagraph{#1}\mbox{}}
\fi
\ifx\subparagraph\undefined\else
\let\oldsubparagraph\subparagraph
\renewcommand{\subparagraph}[1]{\oldsubparagraph{#1}\mbox{}}
\fi

%%% Use protect on footnotes to avoid problems with footnotes in titles
\let\rmarkdownfootnote\footnote%
\def\footnote{\protect\rmarkdownfootnote}

%%% Change title format to be more compact
\usepackage{titling}

% Create subtitle command for use in maketitle
\providecommand{\subtitle}[1]{
  \posttitle{
    \begin{center}\large#1\end{center}
    }
}

\setlength{\droptitle}{-2em}

  \title{MXB262 Workshop week 2: Introduction to R, ggplot2, and Rmarkdown}
    \pretitle{\vspace{\droptitle}\centering\huge}
  \posttitle{\par}
    \author{Dr Kate Helmstedt}
    \preauthor{\centering\large\emph}
  \postauthor{\par}
      \predate{\centering\large\emph}
  \postdate{\par}
    \date{Semester 1, 2021}


\begin{document}
\maketitle

\hypertarget{this-week}{%
\subsection{This Week}\label{this-week}}

This week you will learn how to:

\begin{enumerate}
\def\labelenumi{\arabic{enumi}.}
\setcounter{enumi}{-1}
\tightlist
\item
  Download R, RStudio, and start writing some code
\item
  Use Rmarkdown to make nice and accessible documents with text, code
  chunks, and figures\\
\item
  Set the working directory\\
\item
  Create a simple plot using \texttt{plot()}
\item
  Load R packages
\item
  Use \texttt{ggplot()} to visualise data
\end{enumerate}

Then we will 6. Learn more depth from a great ggplot textbook 7. Do some
twitter sleuthing

\hypertarget{getting-started}{%
\subsection{0.Getting started}\label{getting-started}}

\hypertarget{downloading-what-you-need}{%
\subsubsection{Downloading what you
need}\label{downloading-what-you-need}}

R can be downloaded from \url{http://cran.r‐project.org/} Using a script
editor, such as ``RStudio,'' can also be helpful. RStudio can be
downloaded from \url{http://www.rstudio.com/}

\hypertarget{starting-rstudio}{%
\subsubsection{Starting RStudio}\label{starting-rstudio}}

Click the RStudio icon to open RStudio. The interface is divided into
several panels (clockwise From top left):

\begin{enumerate}
\def\labelenumi{\arabic{enumi}.}
\tightlist
\item
  The source code (supporting tabs)
\item
  The currently active objects/history
\item
  A File browser/plot window/package install window/R help window
  (tabbed)
\item
  The R console The source code editor (top left) is where you type,
  edit and save your R code.
\end{enumerate}

The source code editor (top left) is where you type, edit and save your
R code. The editor supports text highlighting, autocompletes common
functions and parentheses, and allows the user to select R code and run
it in the R console (bottom right) with a keyboard shortcut (Ctrl R on
windows, command-enter on macs). Code will appear in this font:

\begin{Shaded}
\begin{Highlighting}[]
\KeywordTok{mean}\NormalTok{(}\KeywordTok{c}\NormalTok{(}\DecValTok{3}\NormalTok{, }\DecValTok{10}\NormalTok{))}
\end{Highlighting}
\end{Shaded}

\begin{verbatim}
## [1] 6.5
\end{verbatim}

\hypertarget{starting-a-new-script}{%
\subsubsection{Starting a new script}\label{starting-a-new-script}}

Let's open a new script and save it to your harddrive. In matlab, this
is the equivalent to an `m file'. Instead of typing every command into
the console, making a script lets you record and save everything that
you do. So next week if you're wondering how you made that line plot, or
did that calculation, or ran that multi-line simulation, you can just
pull up the R script and run it exactly the same. Remember that
\#dataviz is about reproducibility -- using scripts in R is one step
toward that goal.

When writing R scripts, use lots of \# comments throughout your R
scripts to record what you were thinking or what the code does. Any line
or text that starts with \# won't run, it's just there to communicate
with whoever reads your code later. Usually that is you, but sometimes
it will be your tutors or your collaborators. If you liberally use
comments you will thank yourself when you go back to the analysis later!

Similarly at the start of your script, put some meta-information about
the whole script, such as: \# who wrote the code? \# what does the code
do? \# when did you write the code? etc.

The more projects you work on and analyses you do, the more important it
is to have this meta- information at the beginning of your code. Every
academic and data scientist wishes they were just a little bit better at
doing this, so start now.

\hypertarget{r-markdown}{%
\subsection{1. R Markdown}\label{r-markdown}}

There is an R Markdown document loaded into the Workshop files for this
week. In MXB262 all assignments will be completed in this format.
Markdown is a simple formatting syntax for authoring HTML, PDF, and MS
Word documents. For more details on using R Markdown see
\url{http://rmarkdown.rstudio.com}.

Open the R markdown document. When you click the \textbf{Knit} button
(at the top of the Editor window in RStudio -- it helpfully has a little
ball of wool and knitting needles next to it), a document will be
generated that includes the content, figures, and the output of any
embedded R code chunks within the document. This is useful because then
everything is in one place -- you only need to make one document with
your text, code, and figures, not create a Word doc and continually copy
updated code, results, and figures into it.

The real magic of an R Markdown script is that you can embed an R code
chunk like this:

\begin{Shaded}
\begin{Highlighting}[]
\KeywordTok{plot}\NormalTok{(pressure)}
\end{Highlighting}
\end{Shaded}

In the knitted pdf, you'll see the code (```\{`plot(pressure)'\}), plus
the output from the code (the figure). In the source code (i.e.~the
script), you'll just see the code, but when you press the green `play'
button next to it the figure will show up.

Creating headers, tables, lists and indentations is easy, you can find a
cheat sheet here
\url{https://www.rstudio.com/wp-content/uploads/2015/02/rmarkdown-cheatsheet.pdf}.
Save the cheat sheet to your desktop or print it out so you have it on
hand for the workshops (or do what I do, and google it every single
time).

You're going to be using R Markdown for all worksheets, Problem Solving
tasks 1-6, and the summaries of your projects. If you are having
troubles with R Markdown at any point, please ask your tutor in
practicals or via email, or you won't be able to submit your assessment.

\hypertarget{task}{%
\paragraph{TASK}\label{task}}

\begin{quote}
\begin{enumerate}
\def\labelenumi{\arabic{enumi}.}
\tightlist
\item
  Update the author (your name) and date fields at the top of the R
  markdown file in the Workshop files.
\item
  Use the knit button to render the file to pdf.\\
\item
  Run the code in only the individual chunk in the R markdown file
  instead of the full document by clicking the \emph{Run} button inside
  the chunk (the green arrow to the right of the `````\{r cars\}''
  line). Or, run each line of code one by one in the console by placing
  your cursor inside the chunk and pressing \emph{Cmd+Shift+Enter}.
\end{enumerate}
\end{quote}

\hypertarget{adding-code-chunks-in-r-markdown}{%
\subsubsection{Adding code chunks in R
markdown}\label{adding-code-chunks-in-r-markdown}}

R markdown is great because it allows us to combine natural language
text (like this) and runable R code. The runable R code is written in
chunks, and we need to delineate those in the file so R knows to treat
them as code instead of text.

\hypertarget{task-1}{%
\paragraph{TASK}\label{task-1}}

\begin{quote}
\begin{enumerate}
\def\labelenumi{\arabic{enumi}.}
\setcounter{enumi}{3}
\tightlist
\item
  Add a new code chunk in the R markdown file by clicking the
  \emph{Insert} button on the toolbar in R Studio and then selecting
  \emph{R} or by pressing \emph{Cmd+Option+I} for Mac or
  \emph{ctrl+alt+I} for PC.
\item
  Within that code chunk add a new plot using the \texttt{plot()}
  function, and the dataset \texttt{discoveries}
\end{enumerate}
\end{quote}

\hypertarget{setting-the-working-directory}{%
\subsection{2. Setting the working
directory}\label{setting-the-working-directory}}

\emph{Here, we will leave R markdown and jump back into an R script and
the Console}

To use R you will to access data, save files, and sometimes access
images. The folder that stores all documents for a project is called the
\emph{working directory}. R needs to know where to look for data files
and/or where to save generated files, that is, where the \emph{working
directory} for this project is. Follow the following steps to set-up the
working directory for MXB262.

\begin{enumerate}
\def\labelenumi{\arabic{enumi}.}
\tightlist
\item
  In the console use the \texttt{setwd("add\ file\ path\ here")}
  function to set the working directory to a file of your choice. You
  must keep the " " marks inside the brackets.
\item
  Type the \texttt{getwd()} function in the console to check that
  returned file path is correct.
\end{enumerate}

\textbf{NOTE: Please do not submit any code that contains the
\texttt{setwd()} function. Please comment the line of code out by
placing a \# in front of the code before submitting. If you include this
command, when the tutor is trying to mark your code it will try to
reorganise the files on the tutor's computer. This can make marking very
difficult!}

\hypertarget{data}{%
\subsection{3. Data}\label{data}}

This unit is all about data visualisation, which means we need data. To
make this easy to start with, we will be using data that is already
clean, ready to use and available within R. See (and save)
\href{https://stat.ethz.ch/R-manual/R-devel/library/datasets/html/00Index.html}{this
repository} for a list of all datasets that come pre-loaded into R.

\hypertarget{task-2}{%
\paragraph{TASK}\label{task-2}}

\begin{quote}
\begin{enumerate}
\def\labelenumi{\arabic{enumi}.}
\setcounter{enumi}{4}
\tightlist
\item
  Add a plot of one of the datasets listed on the
  (\href{https://stat.ethz.ch/R-manual/R-devel/library/datasets/html/00Index.html}{R
  data packages} webpage. Use the \texttt{plot()} function that we
  already used in the code chunk above.
\end{enumerate}
\end{quote}

\begin{Shaded}
\begin{Highlighting}[]
\CommentTok{# Your first plot! In these tutorials, you can type directly into this box and press "Run Code". If you need to start over from the original prompt, click "Start Over"}
\KeywordTok{plot}\NormalTok{()}
\end{Highlighting}
\end{Shaded}

\hypertarget{loading-packages}{%
\subsection{4. Loading packages}\label{loading-packages}}

Packages in R can contain functions as well as data. R comes with some
base packages when you first install it, but if you wish to use
additional packages you need to \emph{install} and then \emph{load}
them.

To install a package use the \texttt{install.package("package\_name")}
function (do this on every new computer), and to load the package use
\texttt{library(package\_name)} (do this every time you re-open R).

This code chunk shows how to install and load the \emph{ggplot2}
library. Note that the first line is commented out using \texttt{\#} to
stop the \texttt{install.packages()} function being run every time the
document is rendered. You will need to delete the \texttt{\#} and run
both lines. I suggest you add the \texttt{\#} back in once you have
successfully installed ggplot2.

\begin{Shaded}
\begin{Highlighting}[]
\CommentTok{#install.packages("ggplot2")}
\KeywordTok{library}\NormalTok{(ggplot2)}
\end{Highlighting}
\end{Shaded}

When you are using R directly from the console, you will need to load
ggplot2 (and all other packages) every time by typing the
\texttt{library()} command into the console. In an R markdown file, the
packages are all loaded in the first code chunk, along with the
\texttt{knitr::opts\_chunk\$set(echo\ =\ TRUE)} line. Take a careful
look back at that R markdown file now so you can see where I have placed
it -- you will need to replicate this for each R markdown file you
create, and for each package you will be using.

\hypertarget{visualisation-in-r---ggplot}{%
\subsection{5. Visualisation in R -
ggplot()}\label{visualisation-in-r---ggplot}}

Throughout these practicals we'll be using Hadley Wickham's
\texttt{Grammar\ of\ Graphics} package (more commonly called
\texttt{ggplot}). ggplot has a specific format for creating plots,
graphs and other visualisations. Once you understand the format, making
and customising graphics is intuitive.

You can think of building a plot like building a house, the essential
ingredients are: the building materials (your data) and a floor plan
(called the \texttt{geom} in ggplot). The data is just a useless heap on
its own (like bricks and wood), and a floor plan or \texttt{geom}
achieves nothing without materials. After the house is built you can add
fittings and fixtures like taps, door handles, and the house number
(title, labels, change the colours).

\begin{longtable}[]{@{}lll@{}}
\toprule
\begin{minipage}[b]{0.30\columnwidth}\raggedright
house\strut
\end{minipage} & \begin{minipage}[b]{0.31\columnwidth}\raggedright
plot\strut
\end{minipage} & \begin{minipage}[b]{0.30\columnwidth}\raggedright
ggplot function\strut
\end{minipage}\tabularnewline
\midrule
\endhead
\begin{minipage}[t]{0.30\columnwidth}\raggedright
Building Materials\strut
\end{minipage} & \begin{minipage}[t]{0.31\columnwidth}\raggedright
Your data set and which variables in the dataset to use\strut
\end{minipage} & \begin{minipage}[t]{0.30\columnwidth}\raggedright
\texttt{ggplot(dataset,\ aes(x\ =\ variable,\ y\ =\ variable)}\strut
\end{minipage}\tabularnewline
\begin{minipage}[t]{0.30\columnwidth}\raggedright
Floor plan\strut
\end{minipage} & \begin{minipage}[t]{0.31\columnwidth}\raggedright
what type of \texttt{geom} (graph or plot)\strut
\end{minipage} & \begin{minipage}[t]{0.30\columnwidth}\raggedright
\texttt{+\ geom\_point()}\strut
\end{minipage}\tabularnewline
\begin{minipage}[t]{0.30\columnwidth}\raggedright
Fittings and fixtures\strut
\end{minipage} & \begin{minipage}[t]{0.31\columnwidth}\raggedright
add a title, change axis labels, add colours etc\strut
\end{minipage} & \begin{minipage}[t]{0.30\columnwidth}\raggedright
\texttt{+\ ggtitle()} \texttt{+\ ylab()}, \texttt{+\ xlab()} etc\strut
\end{minipage}\tabularnewline
\begin{minipage}[t]{0.30\columnwidth}\raggedright
Add it all together\strut
\end{minipage} & \begin{minipage}[t]{0.31\columnwidth}\raggedright
\strut
\end{minipage} & \begin{minipage}[t]{0.30\columnwidth}\raggedright
\texttt{ggplot(dataset,\ aes(x\ =\ variable,\ y\ =\ variable)\ +\ geom\_bar()\ +\ ggtitle()}\strut
\end{minipage}\tabularnewline
\bottomrule
\end{longtable}

The following code re-creates the pressure plot above but using ggplot.

\begin{Shaded}
\begin{Highlighting}[]
\CommentTok{# 1. Load ggplot2 library/package}
\KeywordTok{library}\NormalTok{(ggplot2)}

\CommentTok{# 2. The building materials are: ggplot(pressure, aes(x= temperature, y = pressure)) }
\CommentTok{# 3. Define the floor/graph plan (tell ggplot which type of graph to make) : + geom_point()}
\CommentTok{# 4. Add titles and axis labels: ggtitle("Go forth and create awesome data visualisations") + ylab("Take the pressure down") + xlab("It's getting hot in here")}
\CommentTok{# 5. Assign the plot to an object called 'scatterplot_house'}

\KeywordTok{ggplot}\NormalTok{(pressure, }\CommentTok{# your dataset}
       \KeywordTok{aes}\NormalTok{(}\DataTypeTok{x=}\NormalTok{ temperature, }\DataTypeTok{y =}\NormalTok{ pressure)) }\OperatorTok{+}\StringTok{ }\CommentTok{# your chosen variables in the dataset}
\StringTok{  }\KeywordTok{geom_point}\NormalTok{() }\OperatorTok{+}\StringTok{ }\CommentTok{# what kind of plot you want}
\StringTok{  }\KeywordTok{ggtitle}\NormalTok{(}\StringTok{"Go forth and create awesome data visualisations"}\NormalTok{) }\OperatorTok{+}\StringTok{ }\CommentTok{# your title}
\StringTok{  }\KeywordTok{ylab}\NormalTok{(}\StringTok{"Under pressure"}\NormalTok{) }\OperatorTok{+}\StringTok{ }\CommentTok{# your y axis label}
\StringTok{  }\KeywordTok{xlab}\NormalTok{(}\StringTok{"It's getting hot in here"}\NormalTok{) }\CommentTok{# your x axis label}
\end{Highlighting}
\end{Shaded}

A cheat sheet for ggplot can be found
\url{https://www.rstudio.com/wp-content/uploads/2015/03/ggplot2-cheatsheet.pdf}.

\hypertarget{r-for-data-science-exercises}{%
\subsection{6. R for Data Science
exercises}\label{r-for-data-science-exercises}}

We will be using Hadley Wickham's free online book
\href{https://r4ds.had.co.nz/.html}{R for Data Science} to learn more
about ggplot. You can start at Chapter 3 Data Visualisation, however the
introductory chapters may also be useful if you are interested in
starting there outside of class time.

\hypertarget{tasks}{%
\paragraph{TASKS}\label{tasks}}

\begin{quote}
\begin{enumerate}
\def\labelenumi{\arabic{enumi}.}
\setcounter{enumi}{5}
\tightlist
\item
  Work through
  \href{https://r4ds.had.co.nz/data-visualisation.html}{Chapter 3 of R
  for Data Science}. Read and follow all the examples along in your own
  R console for sections:
\end{enumerate}
\end{quote}

\begin{itemize}
\tightlist
\item
  \href{https://r4ds.had.co.nz/data-visualisation.html\#introduction-1}{3.1}
\item
  \href{https://r4ds.had.co.nz/data-visualisation.html\#first-steps}{3.2}
  (including all exercises)
\item
  \href{https://r4ds.had.co.nz/data-visualisation.html\#aesthetic-mappings}{3.3}
  (and exercises 1-3)
\item
  \href{https://r4ds.had.co.nz/data-visualisation.html\#common-problems}{3.4}
\item
  \href{https://r4ds.had.co.nz/data-visualisation.html\#facets}{3.5}
  (skip the exercises)
\item
  \href{https://r4ds.had.co.nz/data-visualisation.html\#geometric-objects}{3.6}
  (skip the exercises).
\end{itemize}

You can go back to the exercises you skipped outside of class time,
although they aren't critical at this stage of the unit.

\hypertarget{trumps-twitter-habits}{%
\subsection{7. Trump's Twitter Habits}\label{trumps-twitter-habits}}

Work through (i.e.~use the code to generate for yourselfin your R
console in R Studio) the first 3 plots in
\href{http://varianceexplained.org/r/trump-tweets/}{Donald Trump's
twitter example}. We talked about this exploration in the lecture, so if
you haven't watched those videos jump back and watch those now.

Although all of the code here is useful, pay particular attention to the
ggplot commands being used here. These are all simple, but well-chosen
and effective plots.

NOTE 1: you will need to install the three packages used first. If you
went right ahead and started following the exercise without reading this
note and got your first error -- congratulations on entering the world
of R programing, this won't be your last error.

NOTE 2: Trump got cancelled on Twitter in 2021! As a result, we won't be
able to download his tweets like they do on the blog in their second
code chunk. Instead, start by downloading the dataset by running the
third code chunk.

R is a constantly evolving language. At some point, the authors of the
\texttt{tidytext} package cleaned up some clunky commands but the blog
hasn't been updated. The old piece of code from the blog

\begin{Shaded}
\begin{Highlighting}[]
\NormalTok{nrc <-}\StringTok{ }\NormalTok{sentiments }\OperatorTok
\StringTok{  }\KeywordTok{filter}\NormalTok{(lexicon }\OperatorTok{==}\StringTok{ "nrc"}\NormalTok{) }\OperatorTok
\StringTok{  }\NormalTok{dplyr}\OperatorTok{::}\KeywordTok{select}\NormalTok{(word, sentiment)}
\end{Highlighting}
\end{Shaded}

can now be replaced with one line:

\begin{Shaded}
\begin{Highlighting}[]
\NormalTok{nrc <-}\StringTok{ }\KeywordTok{get_sentiments}\NormalTok{(}\StringTok{"nrc"}\NormalTok{)}
\end{Highlighting}
\end{Shaded}

Once they start getting into the word analysis, they stop walking us
through the visualisations. Code for those figures follows here, so keep
following and running all of the analysis and data-wrangling on the blog
but jump back to grab this code when you reach each of these questions
and figures:

What were the most common words in Trump's tweets overall?

\begin{Shaded}
\begin{Highlighting}[]
\NormalTok{tweet_words }\OperatorTok
\KeywordTok{count}\NormalTok{(word, }\DataTypeTok{sort =} \OtherTok{TRUE}\NormalTok{) }\OperatorTok
\KeywordTok{head}\NormalTok{(}\DecValTok{20}\NormalTok{) }\OperatorTok
\KeywordTok{mutate}\NormalTok{(}\DataTypeTok{word =} \KeywordTok{reorder}\NormalTok{(word, n)) }\OperatorTok
\KeywordTok{ggplot}\NormalTok{(}\KeywordTok{aes}\NormalTok{(word, n)) }\OperatorTok{+}
\KeywordTok{geom_bar}\NormalTok{(}\DataTypeTok{stat =} \StringTok{"identity"}\NormalTok{) }\OperatorTok{+}
\KeywordTok{ylab}\NormalTok{(}\StringTok{"Occurrences"}\NormalTok{) }\OperatorTok{+}
\KeywordTok{coord_flip}\NormalTok{()}
\end{Highlighting}
\end{Shaded}

Which are the words most likely to be from Android and most likely from
iPhone?

\begin{Shaded}
\begin{Highlighting}[]
\NormalTok{android_iphone_ratios }\OperatorTok
\KeywordTok{group_by}\NormalTok{(logratio }\OperatorTok{>}\StringTok{ }\DecValTok{0}\NormalTok{) }\OperatorTok
\KeywordTok{top_n}\NormalTok{(}\DecValTok{15}\NormalTok{, }\KeywordTok{abs}\NormalTok{(logratio)) }\OperatorTok
\KeywordTok{ungroup}\NormalTok{() }\OperatorTok
\KeywordTok{mutate}\NormalTok{(}\DataTypeTok{word =} \KeywordTok{reorder}\NormalTok{(word, logratio)) }\OperatorTok
\KeywordTok{ggplot}\NormalTok{(}\KeywordTok{aes}\NormalTok{(word, logratio, }\DataTypeTok{fill =}\NormalTok{ logratio }\OperatorTok{<}\StringTok{ }\DecValTok{0}\NormalTok{)) }\OperatorTok{+}
\KeywordTok{geom_bar}\NormalTok{(}\DataTypeTok{stat =} \StringTok{"identity"}\NormalTok{) }\OperatorTok{+}
\KeywordTok{coord_flip}\NormalTok{() }\OperatorTok{+}
\KeywordTok{ylab}\NormalTok{(}\StringTok{"Android / iPhone log ratio"}\NormalTok{) }\OperatorTok{+}
\KeywordTok{scale_fill_manual}\NormalTok{(}\DataTypeTok{name =} \StringTok{""}\NormalTok{, }\DataTypeTok{labels =} \KeywordTok{c}\NormalTok{(}\StringTok{"Android"}\NormalTok{, }\StringTok{"iPhone"}\NormalTok{),}
\DataTypeTok{values =} \KeywordTok{c}\NormalTok{(}\StringTok{"red"}\NormalTok{, }\StringTok{"lightblue"}\NormalTok{))}
\end{Highlighting}
\end{Shaded}

And we can visualize it with a 95\% confidence interval:

\begin{Shaded}
\begin{Highlighting}[]
\KeywordTok{library}\NormalTok{(scales)}
\NormalTok{sentiment_differences }\OperatorTok
\KeywordTok{ungroup}\NormalTok{() }\OperatorTok
\KeywordTok{mutate}\NormalTok{(}\DataTypeTok{sentiment =} \KeywordTok{reorder}\NormalTok{(sentiment, estimate)) }\OperatorTok
\KeywordTok{mutate_each}\NormalTok{(}\KeywordTok{funs}\NormalTok{(. }\OperatorTok{-}\StringTok{ }\DecValTok{1}\NormalTok{), estimate, conf.low, conf.high) }\OperatorTok
\KeywordTok{ggplot}\NormalTok{(}\KeywordTok{aes}\NormalTok{(estimate, sentiment)) }\OperatorTok{+}
\KeywordTok{geom_point}\NormalTok{() }\OperatorTok{+}
\KeywordTok{geom_errorbarh}\NormalTok{(}\KeywordTok{aes}\NormalTok{(}\DataTypeTok{xmin =}\NormalTok{ conf.low, }\DataTypeTok{xmax =}\NormalTok{ conf.high)) }\OperatorTok{+}
\KeywordTok{scale_x_continuous}\NormalTok{(}\DataTypeTok{labels =} \KeywordTok{percent_format}\NormalTok{()) }\OperatorTok{+}
\KeywordTok{labs}\NormalTok{(}\DataTypeTok{x =} \StringTok{"% increase in Android relative to iPhone"}\NormalTok{,}
\DataTypeTok{y =} \StringTok{"Sentiment"}\NormalTok{)}
\end{Highlighting}
\end{Shaded}

We're especially interested in which words drove this different in
sentiment. Let's consider the words with the largest changes within each
category:

\begin{Shaded}
\begin{Highlighting}[]
\NormalTok{android_iphone_ratios }\OperatorTok
\KeywordTok{inner_join}\NormalTok{(nrc, }\DataTypeTok{by =} \StringTok{"word"}\NormalTok{) }\OperatorTok
\KeywordTok{filter}\NormalTok{(}\OperatorTok{!}\NormalTok{sentiment }\OperatorTok\StringTok{ }\KeywordTok{c}\NormalTok{(}\StringTok{"positive"}\NormalTok{, }\StringTok{"negative"}\NormalTok{)) }\OperatorTok
\KeywordTok{mutate}\NormalTok{(}\DataTypeTok{sentiment =} \KeywordTok{reorder}\NormalTok{(sentiment, }\OperatorTok{-}\NormalTok{logratio),}
\DataTypeTok{word =} \KeywordTok{reorder}\NormalTok{(word, }\OperatorTok{-}\NormalTok{logratio)) }\OperatorTok
\KeywordTok{group_by}\NormalTok{(sentiment) }\OperatorTok
\KeywordTok{top_n}\NormalTok{(}\DecValTok{10}\NormalTok{, }\KeywordTok{abs}\NormalTok{(logratio)) }\OperatorTok
\KeywordTok{ungroup}\NormalTok{() }\OperatorTok
\KeywordTok{ggplot}\NormalTok{(}\KeywordTok{aes}\NormalTok{(word, logratio, }\DataTypeTok{fill =}\NormalTok{ logratio }\OperatorTok{<}\StringTok{ }\DecValTok{0}\NormalTok{)) }\OperatorTok{+}
\KeywordTok{facet_wrap}\NormalTok{(}\OperatorTok{~}\StringTok{ }\NormalTok{sentiment, }\DataTypeTok{scales =} \StringTok{"free"}\NormalTok{, }\DataTypeTok{nrow =} \DecValTok{2}\NormalTok{) }\OperatorTok{+}
\KeywordTok{geom_bar}\NormalTok{(}\DataTypeTok{stat =} \StringTok{"identity"}\NormalTok{) }\OperatorTok{+}
\KeywordTok{theme}\NormalTok{(}\DataTypeTok{axis.text.x =} \KeywordTok{element_text}\NormalTok{(}\DataTypeTok{angle =} \DecValTok{90}\NormalTok{, }\DataTypeTok{hjust =} \DecValTok{1}\NormalTok{)) }\OperatorTok{+}
\KeywordTok{labs}\NormalTok{(}\DataTypeTok{x =} \StringTok{""}\NormalTok{, }\DataTypeTok{y =} \StringTok{"Android / iPhone log ratio"}\NormalTok{) }\OperatorTok{+}
\KeywordTok{scale_fill_manual}\NormalTok{(}\DataTypeTok{name =} \StringTok{""}\NormalTok{, }\DataTypeTok{labels =} \KeywordTok{c}\NormalTok{(}\StringTok{"Android"}\NormalTok{, }\StringTok{"iPhone"}\NormalTok{),}
\DataTypeTok{values =} \KeywordTok{c}\NormalTok{(}\StringTok{"red"}\NormalTok{, }\StringTok{"lightblue"}\NormalTok{))}
\end{Highlighting}
\end{Shaded}


\end{document}
